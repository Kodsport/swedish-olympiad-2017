\problemname{Minigolf}

\begin{center}
\includegraphics[width=0.3\textwidth]{minigolfbild.pdf}
\end{center}

En minigolfanläggning har $N$ stycken banor. Varannan bana (udda nummer) är så kallad ``{\em par 2}'' och
varannan (jämna nummer) är ``{\em par 3}'', där ``{\em par}'' är det rekommenderade antalet slag en golfspelare ska klara
en viss bana på. Det finns också en regel som säger att om man slår fler än $7$ slag på en bana räknas det ändå
bara som $7$ slag vid sammanräkningen.

Skriv ett program som, givet antalet slag du använt på varje bana, beräknar det sammanlagda resultatet över/under par. 

\section*{Indata}
Först en rad med ett heltal $N$ $(2\le N\le 10)$, antalet banor. Sedan följer en rad med $N$ heltal mellan $1$ och $10$ vardera: antalet slag du använt på varje bana.

\section*{Utdata}
En rad med ett heltal, det sammanlagda resultatet över/under par.

\section*{Poängsättning}
Din lösning kommer att testas på en mängd testfallsgrupper.
För att få poäng för en grupp så måste du klara alla testfall i gruppen.

\noindent
\begin{tabular}{| l | l | l |}
\hline
  Grupp & Poängvärde & Gränser \\ \hline
  $1$    & $40$       &  Ingen bana tog fler än 7 slag att avklara.  \\ \hline 
  $2$    & $60$       &  Inga ytterligare begränsningar. \\ \hline
\end{tabular}

\section*{Förklaringar till exemplen}

\begin{figure}[!htb]
\minipage{0.5\textwidth}
{\bf Exempel 1:}\\
\begin{tabular}{||l|c|c|c||c||}\hline \hline
Bana & 1 & 2 & 3 & Totalt \\ \hline \hline
Par & 2 & 3 & 2 & 7 \\ \hline 
Slag & 5 & 3 & 1 & 9 \\ \hline
+/- & +3 & 0 & -1 & +2 \\ \hline \hline
\end{tabular}
\endminipage\hfill
\minipage{0.5\textwidth}
{\bf Exempel 2:}\\
\begin{tabular}{||l|c|c|c|c|c|c||c||}\hline \hline
Bana & 1 & 2 & 3 & 4 & 5 & 6 & Totalt \\ \hline \hline
Par & 2 & 3 & 2 & 3 & 2 & 3 & 15 \\ \hline 
Slag & 1 & {\bf 7} & 1 & 1 & 1 & 1 & 12 \\ \hline
+/- & -1 & +4 & -1 & -2 & -1 & -2 & -3\\ \hline \hline
\end{tabular}\\
(observera att de 9 slagen på andra banan bokförs som 7)
\endminipage\hfill
\end{figure}
