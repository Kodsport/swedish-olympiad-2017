\problemname{Armstöd}

Petitess-organisationen (PO) har möte och de $N$ medlemmarna sitter på
stolar i en ring, vända inåt. Mellan varje par av stolar finns ett armstöd som högst en av personerna kan använda. Varje person har en preferens i form av vilken eller vilka armar hen vill placera på armstöden:
\begin{itemize}
\item \texttt{V}: vänster arm
\item \texttt{H}: höger arm
\item \texttt{A}: antingen vänster eller höger arm
\item \texttt{B}: båda armarna
\item \texttt{I}: ingen arm
\end{itemize}

Givet en sträng som beskriver alla $N$ medlemmars preferenser, skriv ett program som beräknar hur många av personerna som maximalt kan bli nöjda.

\section*{Indata}
På första raden står ett heltal $N$ ($5\le N \le 30$), antal personer i ringen. På andra raden står personernas preferenser, givna i den ordning
personerna sitter, {\em moturs} i ringen, i form av en
sträng bestående av $N$ bokstäver, vardera \texttt{V}, \texttt{H},
\texttt{A}, \texttt{B} eller \texttt{I}. 

\section*{Utdata}
Programmet ska skriva ut ett heltal: det maximala antalet personer som kan få sin preferens uppfylld.

\begin{figure}[!htb]
\begin{center}
\includegraphics[width=5cm]{armstodbild.pdf}
\end{center}
\caption{Figuren visar lösningen till exempel $1$. De tjocka linjerna markerar
armar som personer lagt på armstöden. Den grå färgen visar vilka som fått sina preferenser uppfyllda. Pilen markerar var den givna
indatasträngen börjar och slutar.}
\end{figure}

\section*{Poängsättning}
Din lösning kommer att testas på en mängd testfallsgrupper.
För att få poäng för en grupp så måste du klara alla testfall i gruppen.
\noindent
\begin{tabular}{| l | l | l |}
\hline
  Grupp & Poängvärde & Gränser \\ \hline
  $1$    & $20$       &  $N=5$ och första bokstaven är \texttt{I}.\\ \hline 
  $2$    & $40$       &  $5\le N\le 15$ , vilka bokstäver som helst.\\ \hline 
  $3$    & $20$       &  $5\le N\le 30$ och första bokstaven är \texttt{I}.\\ \hline 
  $4$    & $20$       &  Inga ytterligare begränsningar. \\ \hline
\end{tabular}
