\problemname{Datorköp}

Ett företag har precis köpt in nya datorer: $x$ stationära och $y$
bärbara. En stationär dator kostar $a$ dollar och en bärbar $b$
dollar, där $1\le a,b \le 1000$. Företaget har $n$ avdelningar och de
anses olika viktiga gentemot varandra. VDn har bestämt att datorerna
ska distribueras enligt följande enkla regel: {\em En viktigare avdelning ska få datorer till ett värde som är minst lika stort som en mindre viktig avdelning.}

Klara får i uppdrag att göra fördelningen. Trots att Klara är jätteduktig jobbar hon på den minst viktiga avdelningen. För att bli populär bland avdelningskollegorna vill hon förstås ordna datorer till högsta möjliga värde till sin avdelning. 
Hon har bett dig om hjälp! Skriv ett program som, givet variablerna $x$, $a$, $y$, $b$ och $n$ beräknar det högsta möjliga värdet på datorerna som Klaras avdelning kan få?


\section*{Indata}
En rad med de fem heltalen $x$, $a$, $y$, $b$ och $n$.

\section*{Utdata}

Programmet ska skriva ut en rad med ett heltal: det högsta möjliga värdet på datorerna som ges till den minst viktiga avdelningen.

\section*{Poängsättning}
\begin{tabular}{| l | l | l |}
    \hline
      Grupp & Poängvärde & Gränser \\ \hline
      $1$    & $20$       &  $n=2$ och $0\le x,y \le 100$.\\ \hline 
      $2$    & $20$       &  $3\le n\le 6$ och $0\le x,y \le 10$. \\ \hline 
      $3$    & $40$       &  $10\le n\le 100$ och $0\le x,y \le 100$.\\ \hline 
      $4$    & $20$       &  $800\le n\le 1000$ och $0\le x,y \le 1000$. \\ \hline
\end{tabular}
