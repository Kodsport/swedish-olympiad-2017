\problemname{Bilda ord}

Fatimeh studerar sitt hemspråk som använder det arabiska skriftspråket. Just nu
sitter hon med en övning att bilda ord från givna bokstäver. Hon vill veta på
hur många sätt hon kan bilda ett ord med givna bokstäver.

I svenskans skriftspråk skulle övningen kunna se ut så hår "l s å t", du får
alltså 4 givna bokstäver. Fatimeh vet då att hon måste testa 4! == 24
permutationer. Skulle istället bokstäverna vara "l S å t", alltså med ett stort
"S", då vet vi att vi måste placera bokstaven "S" i meningens början. Med det
villkoret är antalet permutationer 3! == 6.

\section*{Indata}

Med bokstäverna i det arabiska skriftspråket har man mycket mer kontext om var i ordet bokstaven kan förekomma, även i relation till andra bokstäver. Fatimeh har skapat en självförklarande notation som specifierar restriktionerna. Om det exempelvis finns 5 stycken tecken (i denna uppgift är dem alltid olika tecken), så kan exempelvis följande restriktioner finnas:

    Bokstav 2 måste va på någon utav platserna [1, 4, 5] (a)
    Bokstav 3 måste föregå någon utav följande bokstäver [2, 4] (b)

För att indatat ska vara enkelt att läsa så skriver man då indatat såhär:

5
a 2 1 4 5
b 3 2 4


\section*{Utdata}

TODO
% Skriv ut en rad med ett heltal. Heltalet är hur mycket njutningspoäng Farah som
% mest kan få med den bästa ätstrategin.

\section*{Poängsättning}

TODO
% För testfall värda upp till $3$ poäng, kommer $N$ vara som mest 5. För full
% poäng så ska ditt program klara $N$ som mest 15.
