\problemname{Sidnumrering}

\begin{center}
\includegraphics[width=0.3\textwidth]{bokbild.pdf}
\end{center}
 
I en viss bok med $N$ sidor vill förlaget spara pengar genom att enbart trycka
sidnummer på höger sida, d.v.s. de udda talen. Skriv ett program som
räknar ut hur många siffror som går åt av varje sort.

\section*{Indata}
Den enda raden innehåller heltalet $N$ ($1 \leq N \leq 10^12$).

\section*{Utdata}
Skriv ut en rad med $10$ heltal, antalet siffror av varje sort. Först antalet nollor, sedan antalet ettor, o.s.v.


\section*{Poängsättning}
Din lösning kommer att testas på en mängd testfallsgrupper.
För att få poäng för en grupp så måste du klara alla testfall i gruppen.

\noindent
\begin{tabular}{| l | l | l |}
\hline
  Grupp & Poängvärde & Gränser \\ \hline
  $1$    & $60$       &  $1 \le N \le 100\,000$   \\ \hline 
  $2$    & $40$       &  Inga ytterligare begränsningar. \\ \hline
\end{tabular}


\section*{Förklaring till exempel 1}

Sidnumren som skrivs är $1$, $3$, $5$, $7$, $9$, $11$, $13$, $15$,
$17$, $19$, $21$ och $23$. Alltså behövs det inga nollor, 8 ettor, 2 tvåor, 3
treor, inga fyror, 2 femmor, inga sexor, 2 sjuor, inga åttor och 2 nior.


