\problemname{Ninety-nine}

You and your friend are playing a game which you call \emph{Ninety-nine}.
You start, by saying either the number $1$ or $2$.
You then take turns, starting with your friend, increasing this number by either $1$ or $2$ in each step.
The first player who gets to say the number $99$ wins!

Write a program which plays this game for you and wins.

\section*{Interactivity}
This problem is interactive.

Your program should start by printing either the number $1$ or $2$ on a single line.
The grader will then read this number (call it $x$), and in return print a line with either $x+1$ or $x+2$, which can be read by your program.
Your program should then print a number which is $1$ or $2$ higher, and so on.

If you manage to win the game and print $99$, your program should exit normally (with status code 0), without printing anything further.
On the other hand, if the grader prints $99$ before you, or if your program prints an invalid value (including numbers larger than $99$), the test group will get judged as Wrong Answer.

You \emph{must} make sure to flush standard output before reading the grader's response, or else your program
will get judged as Time Limit Exceeded. This works as follows in various languages:
\begin{itemize}
  \item Java: \texttt{System.out.println()} flushes automatically.
  \item Python: \texttt{print()} flushes automatically.
  \item C++: \texttt{cout << endl;} flushes, in addition to writing a newline. If using printf, \texttt{fflush(stdout)}.
  \item Pascal: \texttt{Flush(Output)}.
\end{itemize}

\section*{Constraints}
Your solution will be tested on a set of test groups, each worth a number of points.
Each test group contains a set of test cases.
To get the points for a test group you need to solve all test cases in the test group.
Your final score will be the maximum score of a single submission.

\noindent
\begin{tabular}{| l | l | l |}
\hline
Group & Points & Constraints \\ \hline
1     & 30     & Your friend always increases the number by 1. \\ \hline
2     & 30     & Your friend always increases the number by 2 (unless at 98). \\ \hline
3     & 40     & Your friend plays randomly, with each option having 50\% probability (unless at 98). \\ \hline
\end{tabular}
