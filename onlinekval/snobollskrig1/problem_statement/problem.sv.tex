\problemname{Snöbollskrig 1}

I en bågviktad dubbelriktad graf med $N$ noder leker $L$ länder snöbollskrig under IOI. I
början har varje land ett fort i någon nod.

Vid tid $0$ börjar varje land ge sig ut på snöbollskrig. Snöbollskrig fungerar såhär:

\begin{itemize}
\item Om ett land äger ett fort beger sig de tävlande ut längs alla kanter från fortet, med hastighet 1/s.
\item Om två länder möts längs en kant stannar länderna och krigar.
\item Om två länder möts i en nod stannar länderna och krigar.
\item Om ett land når en nod före någon annan bygger det landet ett fort i noden.
\end{itemize}

Avgör vilja par av länder som kommer kriga mot varandra.

\section*{Indata}

Den första raden innehåller heltalet $2 \le N \le 200'000$, mellanslag, och
därefter $2 \le L \le 50$ och sist $M < 500'000$ som är antalet bågar i
grafen. Därefter följer $L$ rader med ett heltal som säger vilken nod deras
startbas är på. Därefter följer $M$ rader med tre heltal, de två första talen
beskriver vilka två olika noder som berörs och det tredje talet anger vikten på
bågen.

Allt i denna uppgift är 0-indexerat. Både noders och länders index.

\begin{lstlisting}
        N L M
        index_startbas_1
        ...
        index_startbas_L
        nod_a_1 nod_b_1 vikt_1
        ...
        nod_a_M nod_b_M vikt_M
\end{lstlisting}

\section*{Utdata}

Du ska skriva ut en lista med vilka länder som krigar mot varandra. Du skriver
varje par på en ny rad, med ett mellanslag som separator, skriv det land med
lägre index först. Du skall även skriva ut dem i sorterad ordning. Eftersom det
är två tal som man sorterar efter säger vi att man måste sortera efter den
första siffran först. 

\section*{Förklaring av exempel 1}

I detta exempel är grafen cyklisk och nästan helt symmetrisk. I fallet mellan
land 0 och land 3 kommer kriget utspelas på bågen, medan de 3 andra krigen
kommer utspelas på en nod.

\section*{Poängsättning}

Din lösning kommer att testas på en mängd testfallsgrupper. För att få poäng för en grupp
så måste du klara alla testfall i gruppen.

\begin{tabular}{| l | l | l | l |}
\hline
Grupp & Poängvärde & Situation \\ \hline
1     & 30         & Grafen är rak som en linje. Alltså utan cykler. \\ \hline
2     & 30         & Inga krig förekommer på noderna. \\ \hline
3     & 40         & Inga begränsningar. \\ \hline
\end{tabular}
