\problemname{IOI-uttagning}
Programmeringsolympiadens arrangörsgrupp vill ut och leka i snön.
Tyvärr har man som arrangör otaliga plikter, som kommer före lekandet.
Man måste göra uppgifter, övervaka tävlingar, åka som ledare, och beräkna vilka deltagare som är uttagna till landslaget utifrån årets resultat.

Det slog dock arrangörerna att den sista biten kan ju någon annan ordna.
T.ex. du.

Reglerna för uttagning till landslaget går till som följer.
Totalt finns det fyra tävlingar som ligger till grund för uttagning; finaltävlingen samt tre s.k. KATT-tävlingar (KATT1, KATT2 och KATT3).

Efter alla tävlingar räknas resultaten samman, och de fyra tävlanden med bäst poäng blir uttagna till International Olympiad in Informatics, IOI.
Utöver IOI ska även ett lag väljas till Baltic Olympiad in Informatics, BOI.
Detta lag består dels av IOI-laget, samt de två bästa tävlanden som går i \emph{ettan eller tvåan} och inte kvalificerade sig till IOI.

När resultaten räknas samman väljs de två bästa resultaten från de tre KATT-tävlingarna ut, samt resultatet från finalen, och summeras.

Inom en tävling räknas resultatet ut genom att \emph{normalisera} poängen en tävlande fick.
Först beräknar man den bästa poängen bland alla deltagare $MAX$, samt medianpoängen bland alla deltagare $MED$.
Om en deltagare fick $x \le MED$ poäng blir deltagarens resultat $50 \cdot \frac{x}{MED}$.
Om en deltagare fick $MED \le x \le MAX$ poäng blir deltagarnas resultat $50 + 50 \cdot \frac{x - MED}{MAX - MED}$.
Detta innebär att resultatet ökar linjärt från 0 till 50 när poängen ökar från 0 till $MED$,
och från 50 till 100 när poängen ökar från $MED$ till $MAX$.

För varje deltagare i tävlingen kommer du får deltagarens resultat i de fyra tävlingarna, samt deltagarens årskurs. Avgör vilka som blir uttagna till IOI och BOI.

\section*{Indata}
Den första raden innehåller ett heltal $6 \le N \le 40$, antalet deltagare.

De $N$ nästa raderna beskriver en deltagare och dess resultat.
En rad börjar med deltagarens namn, som består av upp till 20 små bokstäver från $a$ till $z$, utan mellanslag.
Därefter följer ett tal som är antingen $7, 8, 9, 1, 2, 3$ och beskriver deltagarens årskurs. Årskurserna $7, 8, 9$ representerar här högstadieelever.
Slutligen kommer fyra heltal $0 \le F, K1, K2, K3 \le 700$ - poängen från finalen, KATT1, KATT2 och KATT3 respektive.

I en tävling kommer medianen och maxpoängen alltid vara skiljda från varandra.
Dessutom kommer två deltagare \emph{som blir uttagna} aldrig ha samma sammanslagna resultat.
Det kommer alltid kunna väljas ett lag enligt reglerna ovan.

\section*{Utdata}
På den första raden ska du skriva ut namnen på de fyra deltagare som blev uttagna till IOI.

Nästa rad ska innehålla två namn - de ytterligare deltagare som kvalificerade sig till BOI.

På varje rad ska namnen du skriver ut vara sorterade i bokstavsordning.

\section*{Poängsättning}
Din lösning kommer att testas på en mängd testfallsgrupper. För att få poäng för en grupp så måste du klara alla testfall i gruppen.

\begin{tabular}{| l | l | l | l |}
\hline
Grupp & Poängvärde & Begränsningar \\ \hline
1     & 23         & Medianpoängen i en tävling är alltid 50, och maxpoängen 100. \\ \hline
2     & 13         & Alla elever är förstaårselever \\ \hline
3     & 31         & Resultaten i de tre KATT-tävlingarna var alltid samma. \\ \hline
4     & 33         & Inga begränsningar \\ \hline
\end{tabular}
