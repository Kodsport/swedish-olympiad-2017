\problemname{TODO}
TODO

\section*{Indata}
Den första raden innehåller ett heltal $1 \le N \le 100\,000$: antalet finländare.
Därefter följer $N$ rader, en per finländare.
Varje rad innehåller fyra heltal $a, b, c, t$ ($-10^9 \le a,b,c \le 10^9, 1 \le t \le 100\,000$), vilket representerar att finländaren har glädjefunktion $ax^2 + bx + c$, and bara klarar av temperaturer mellan $0$ och $t$, inklusive.
Funktionen garanteras vara positiv överallt mellan $0$ och $t$.

\section*{Utdata}
Skriv ut ett enda tal: den största möjliga lycka som kan uppnås om temperaturen sätts rätt.
Talet ska skrivas ut med precision minst $10^{-5}$.

\section*{Poängsättning}
\begin{tabular}{| l | l | l |}
\hline
Grupp & Poängvärde & Situation \\ \hline
1     & 25         & $N \le 1000$, alla $t$ är samma. \\ \hline
2     & 25         & $N \le 1000$, alla $a$ är positiva (d.v.s., finnarna gillar temperaturextremer). \\ \hline
3     & 25         & $N \le 1000$, alla $a$ är negativa. \\ \hline
4     & 25         & Inga begränsningar. \\ \hline
\end{tabular}
