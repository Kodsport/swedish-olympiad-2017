\problemname{Julklappsköp}
Snälla Allnäs ska köpa \emph{en} julklapp till \emph{vardera} sina $K$ vänner (trots att det är februari -- Allnäs tror på att ha god marginal).
Butiken hon är i har exakt ett exemplar av varje vara.
Det finns totalt $N$ varor.
Allnäs känner sina vänner mycket bra -- hon vet exakt vem som gillar vad och hur mycket.
Hon har skrivit ner en lista med alla $a_{ij}$ tal, talen som säger hur mycket vän $i$ gillar present $j$.

Nu vill Allnäs maximera sina vänners glädje.
Hon vill ge sina vänner presenter på ett sånt sätt, att summan av glädjen för varje vän (d.v.s talen $a_{ij}$) blir maximal.
Vilka julklappar ska hon köpa för att maximera summan av sina vänners glädje?

\section*{Indata}
Den första raden innehåller två heltal $K$ (antal vänner) och $N$ (antal rader).

De följande $K$ raderna innehåller $N$ heltal vardera.
På den $i$:te raden är det $j$:te heltalet $0 \le a_{ij} \le 10^8$ -- hur glad den $i$:te vännen gillar om den får den $j$:te presenten.

\section*{Utdata}
Du ska skriva ut ett heltal -- den maximala summan av vännernas glädje.

\section*{Poängsättning}
Din lösning kommer att testas på en mängd testfallsgrupper. För att få poäng för en grupp
så måste du klara alla testfall i gruppen.

\begin{tabular}{| l | l | l | l |}
\hline
Grupp & Poängvärde & Gränser & Övrigt \\ \hline
\end{tabular}

\section*{Förklaring av exempel 1}
Om Allnäs köper present 3 till vän 1 och present 2 till vän 2 blir summan $a_{31} + a_{22} = 4 + 7 = 11$, vilket är det bästa som går.
