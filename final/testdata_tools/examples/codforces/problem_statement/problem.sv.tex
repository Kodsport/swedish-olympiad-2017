\def\version{jury-2}
\problemname{Kodkraft}
\newcommand{\cf}{kodkraft\texttrademark{}}
Nicolas vill börja tävla i programmering på hemsidan \cf.
Det finns jättemånga olika divisioner man kan tävla i, men eftersom Nicolas är en ny deltagare på \cf så måste han börja i den lägsta divisionen (division 1).
Nicolas mål är att så snabbt som möjligt komma upp till högsta divisionen (division $K$) och vinna en tävling i den.

Enligt kodkrafts\texttrademark{} regler får man bara gå upp en division per tävling, så han kommer behöva göra minst en tävling i varje division.
Nicolas är dock väldigt självsäker och tror därför att han kommer behöva göra exakt en tävling i varje division för att gå upp till nästa division.
När det är tävling på \cf så är det bara en division i taget som tävlar, och två tävlingar överlappar aldrig i tiden.
Tävlingarna följer dessutom samma schema varje år.

Nicolas får påbörja sitt tävlande på \cf vilket datum på året han vill.
Det Nicolas menar med så snabbt som möjligt är att så få tävlingar som möjligt ska gå på \cf (oavsett om han deltar i dessa eller inte) mellan den första tävling han deltar i, och den första vinsten Nicolas har i den högsta divisionen.
Hjälp Nicolas att beräkna hur många tävlingar som krävs!

\section*{Indata}
Den första raden innehåller två heltal $N$ och $K$ ($1 \leq K \leq N \leq 10^6$), antalet tävlingar per år, samt antalet divisioner.

Därefter kommer en rad med $N$ heltal $x_1, \dots, x_N$, ($1 \leq x_i \leq K$), schemat för tävlingarna under ett år.
$x_i$ är divisionen som tävlar under den $i$:te tävlingen efter nyår.
Varje division mellan $1$ och $K$ har minst en tävling under året.

\section*{Utdata}
Ett heltal, det minsta antalet tävlingar som behöver gå på \cf från det att han börjar tävla där tills han har vunnit division $K$.

\section*{Poängsättning}
Din lösning kommer att testas på en mängd testfallsgrupper.
För att få poäng för en grupp så måste du klara alla testfall i gruppen.

\noindent
\begin{tabular}{| l | l | l |}
\hline
Grupp & Poängvärde & Gränser \\ \hline
$1$    & $15$         & $N \leq 10^2$ \\ \hline
$2$    & $20$         & $N \leq 10^3$ \\ \hline
$3$    & $25$         & $N = K$ \\ \hline
$4$    & $40$         & Inga ytterligare begränsningar\\ \hline
\end{tabular}

\section*{Förklaring av exempelfall 3}
Det snabbaste sättet för Nicolas att nå sitt mål är genom att låta sin första tävling vara den andra för division 1 under året.
Sen väntar han i fyra tävlingar för att sedan delta i den första tävlingen för division 2 året efter.
Sen väntar han tre tävlingar och tävlar i division 3.
Sen väntar han i fem tävlingar och tävlar i division 4.
Sen väntar han i 2 tävlingar för att vinna division 5.
Totalt krävs det $1+(4+1)+(3+1)+(5+1)+(2+1) = 19$ tävlingar.
